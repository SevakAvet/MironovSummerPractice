\documentclass[10pt]{article}
\usepackage{csit-2014}

\begin{document}
\title
    {Образец статьи на~конференцию КНИТ-2014}
\author
    {Автор~И.\,О.\authorrefmark{1}, Соавтор~И.\,О.\authorrefmark{2}, Фамилия~И.\,О.\authorrefmark{3}}
\thanks
    {Работа выполнена при финансовой поддержке РФФИ, проект \No\,00-00-00000.}
\email
    {\authorrefmark{1}author1@site.ru, \authorrefmark{2}author2@site.ru, \authorrefmark{3}author3@site.ru}
\organization
    {\authorrefmark{1}Организация, Город, Страна; \authorrefmark{2}\authorrefmark{3}Организация, Город, Страна}
% если автор у статьи один, то команда \authorrefmark{1} не нужна.     
    
\abstract
    {Данный текст является образцом оформления статьи, подаваемой на конференцию КНИТ-2014.
    Аннотация кратко характеризует основную цель работы,
    особенности предлагаемого подхода и~основные результаты.}
\keywords
    {образец, пример, оформление}

\maketitle

\section*{Введение}
После аннотации, но перед первым разделом,
может идти неформальное введение,
описание предметной области,
обоснование актуальности задачи,
краткий обзор известных результатов,
и~т.\,п. В любом случае, структура статьи остается прерогативой авторов.

\section{Название раздела}
Данный документ демонстрирует оформление статьи,
подаваемой на международную конференцию
<<Компьютерные науки и информационные технологии>> \mbox{КНИТ-2014}.
Более подробные инструкции по~стилевому файлу \texttt{csit-2014.sty}
и~использованию издательской системы \LaTeXe\
находятся в~документе \texttt{authors-guide.pdf}.
Работу над статьёй удобно начинать с~правки \TeX-файла данного документа.
Допустимый объём статьи "--- от полутора до четырёх страниц формата А4. 
Необходимость увеличения объема публикации необходимо согласовывать 
с оргкомитетом конференции (\url{knit2014@sgu.ru})

\paragraph{Название параграфа.}
%Первый раздел может содержать формальную постановку задачи,
%основные определения и~обозначения,
%известные факты, необходимые для понимания основных результатов работы,
%и~т.\,п.
Нет никаких ограничений на~количество разделов и~параграфов в~статье.

\paragraph{Теоретическую часть работы}(если таковая имеется) желательно структурировать
с~помощью окружений
Def, Axiom, Hypothesis, Problem, Lemma, Theorem, Corollary, State, Example, Remark.

\begin{Def}
    Математический текст \emph{хорошо структурирован},
    если в~нём выделены определения, теоремы, утверждения, примеры, и~т.\,д.,
    а~неформальные рассуждения (мотивации, интерпретации)
    вынесены в~отдельные параграфы.
\end{Def}

\begin{State}
    Мотивации и~интерпретации наиболее важны для понимания сути работы.
\end{State}

\begin{Theorem}
    Не~менее $90\%$ коллег, заинтересовавшихся Вашей статьёй,
    прочитают в~ней не~более~$10\%$ текста,
    причём это будут именно те~разделы, которые не содержат формул.
\end{Theorem}

\begin{Remark}
    Выше показано применение окружений
    Def, Theorem, State, Remark.
\end{Remark}

\section{Некоторые формулы}

Образец формулы: $f(x_i,\alpha^\gamma)$.

Образец выключной формулы без номера:
\[
    y(x,\alpha) =
    \begin{cases}
        -1, & \text{если } f(x,\alpha)<0;  \\
        +1, & \text{если } f(x,\alpha)\geq 0.
    \end{cases}
\]

Образец выключной формулы с номером:
\begin{equation}
\label{eq:cases}
    y(x,\alpha) =
    \begin{cases}
        -1, & \text{если } f(x,\alpha)<0;  \\
        +1, & \text{если } f(x,\alpha)\geq 0.
    \end{cases}
\end{equation}

Образец выключной формулы, разбитой на две строки с~помощью окружения align:
\begin{align}
    R'_N(F)
        = \frac1N \sum_{i=1}^N
        \Bigl(
            & P(+1\cond x_i) C\bigl(+1,F(x_i)\bigr)+{}
        \notag % подавили номер у первой строки
    \\ {}+{}
            & P(-1\cond x_i) C\bigl(-1,F(x_i)\bigr)
        \Bigr).
        \label{eq:R(F)}
\end{align}

Образцы ссылок: формулы~\eqref{eq:cases} и~\eqref{eq:R(F)}.

\section*{Заключение}
Если этот раздел присутствует, то он не~должен дословно повторять аннотацию.
Обычно здесь отмечают,
каких результатов удалось добиться,
какие проблемы остались открытыми.

\begin{thebibliography}{1}
\bibitem{author09anyscience}
    \BibAuthor{Author\;N.}
    \BibTitle{Paper title}~//
    10-th Int'l. Conf. on Anyscience, 2009.~--- Vol.\,11, No.\,1.~--- Pp.\,111--122.
\bibitem{myHandbook}
    \BibAuthor{Автор\;И.\,О.}
    Название книги.~---
    Город: Издательство, 2009.~--- 314~с.
\bibitem{author09first-word-of-the-title}
    \BibAuthor{Автор\;И.\,О.}
    \BibTitle{Название статьи}~//
    Название конференции или сборника,
    Город:~Изд-во, 2009.~--- С.\,5--6.
\bibitem{author-and-co2007}
    \BibAuthor{Автор\;И.\,О., Соавтор\;И.\,О.}
    \BibTitle{Название статьи}~//
    Название журнала.~--- 2007.~--- Т.\,38, \No\,5.~--- С.\,54--62.
\bibitem{bibUsefulUrl}
    \BibUrl{www.site.ru}~---
    Название сайта~--- 2007.
\bibitem{voron06latex}
    \BibAuthor{Воронцов~К.\,В.}
    \LaTeXe\ в~примерах.~---
    2006.~---
    \BibUrl{http://www.ccas.ru/voron/latex.html}.
\bibitem{Lvovsky03}
    \BibAuthor{Львовский~С.\,М.} Набор и вёрстка в пакете~\LaTeX.~---
    3-е издание.~---
    Москва:~МЦHМО, 2003.~--- 448~с.
\end{thebibliography}

\end{document}
